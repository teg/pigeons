\documentclass[a4paper,10pt,draft]{article}

\usepackage{times}
\usepackage[T1]{fontenc}

\usepackage{a4wide}

\usepackage{ucs}
\usepackage[utf8x]{inputenc}
\usepackage{amsmath}
\usepackage[noxy]{virginialake}

\usepackage[pdftex,pdfborder={0 0 0}]{hyperref}

\usepackage{amsthm}
\theoremstyle{plain}
\newtheorem{theorem}{Theorem}
\newtheorem{lemma}[theorem]{Lemma}
\newtheorem{proposition}[theorem]{Proposition}
\newtheorem{corollary}[theorem]{Corollary}

\theoremstyle{definition}
\newtheorem{definition}[theorem]{Definition}
\newtheorem{remark}[theorem]{Remark}
\newtheorem{notation}[theorem]{Notation}
\newtheorem{para}[theorem]{}

\title{Polynomial Proof of the Pigeon Hole Principle}

\author{Lutz and Tom}

\begin{document}

\maketitle

\newcommand{\fff  }{{\mathsf{f}}}
\newcommand{\ttt  }{{\mathsf{t}}}
\newcommand{\ai   }{{\mathsf{ai}}}
\newcommand{\aw   }{{\mathsf{aw}}}
\newcommand{\ac   }{{\mathsf{ac}}}
\newcommand{\aid  }{{\ai{\downarrow}}}
\newcommand{\awd  }{{\aw{\downarrow}}}
\newcommand{\acd  }{{\ac{\downarrow}}}
\newcommand{\aiu  }{{\ai{\uparrow}}}
\newcommand{\awu  }{{\aw{\uparrow}}}
\newcommand{\acu  }{{\ac{\uparrow}}}
\newcommand{\swi  }{\mathsf{s}}
\newcommand{\med  }{\mathsf{m}}
\newcommand{\asor }{{=_\mathsf{a}{\downarrow}}}
\newcommand{\asand}{{=_\mathsf{a}{\uparrow}}}
\newcommand{\coor }{{=_{\vee\mathsf{c}}}}
\newcommand{\coand}{{=_{\wedge\mathsf{c}}}}
\newcommand{\fffd }{{{=_{\fff}}{\downarrow}}}
\newcommand{\fffu }{{{=_{\fff}}{\uparrow}}}
\newcommand{\tttd }{{{=_{\ttt}}{\downarrow}}}
\newcommand{\tttu }{{{=_{\ttt}}{\uparrow}}}
\newcommand{\tttord }{{{=_{\ttt\vee}}{\downarrow}}}
\newcommand{\fffandd }{{{=_{\fff\wedge}}{\downarrow}}}
\newcommand{\tttoru }{{{=_{\ttt\vee}}{\uparrow}}}
\newcommand{\fffandu }{{{=_{\fff\wedge}}{\uparrow}}}

\newcommand{\AND}[2]{\bigwedge_{#1}^{#2}}
\newcommand{\OR}[2]{\bigvee_{#1}^{#2}}

\newcommand{\Count}{\mathsf{Count}}
\newcommand{\PHP}[1]{\mathsf{PHP}_{#1}}

\begin{remark}
We will encode numbers by sequences of formulae. When we define operations on numbers, we expect the inputs to have the same number of bits, and the output will increase the number of bits if and only if this is necessary to ensure there is no overflow.
\end{remark}

\newcommand{\Add}{\mathsf{Add}}

\begin{lemma}\label{lem:basics}
Some basic results
\begin{enumerate}
%  \item
% \[
% \vlproof{}{}{\Add(a,b)=\Add(b,a)}
% \quad,\]
% \item
% \[
% \vlder{}{}{\Add(a_1,a_2)<\Add(b_1,b_2)}{\vls(a_1\leq b_1.a_2< b_2)}
% \]
 \item
\[
\vlder{}{}{\vls(a>b.b=c)}{a>c}
\]
 \item
\[
\vlder{}{}{\vls(b>a.b=c)}{c>a}
\quad,\quad\text{and}\]
 \item
\[
\vlder{}{}{\vls[a_1> b_1.a_2> b_2]}{\Add(a_1,a_2)>\Add(b_1,b_2)}
\quad.
\]
\end{enumerate}
\end{lemma}

\newcommand{\AddCarryF}{\mathsf{AddCarry4}}
\newcommand{\AddSaveF}{\mathsf{AddSave4}}
\newcommand{\AddCarry}{\mathsf{AddCarry}}
\newcommand{\AddSave}{\mathsf{AddSave}}
\newcommand{\CountCarry}{\mathsf{CountCarry}}
\newcommand{\CountSave}{\mathsf{CountSave}}
\begin{definition}
Definitions of some predicates:
\begin{itemize}
 \item $\AddCarryF(\vec a,\vec b,\vec c,\vec d)=\AddCarry(\AddCarry(\vec a,\vec b,\vec c),\AddSave(\vec a,\vec b,\vec c),\vec d)$,
 \item $\AddSaveF(\vec a,\vec b,\vec c,\vec d)=\AddSave(\AddCarry(\vec a,\vec b,\vec c),\AddSave(\vec a,\vec b,\vec c),\vec d)$,
 \item $\CountCarry(a_1)=\fff$,
 \item $\CountCarry(a_1,\dots,a_{2^\rho})=\AddCarryF(\CountCarry(a_1,\dots,a_{2^{\rho-1}}),\CountSave(a_1,\dots,a_{2^{\rho-1}}),\CountCarry(a_{2^{\rho-1}+1},\dots,a_{2^\rho}),\CountSave(a_{2^{\rho-1}+1},\dots,a_{2^\rho}))$,
 \item $\CountSave(a_1)=a_1$,
 \item $\CountSave(a_1,\dots,a_{2^\rho})=\AddSaveF(\CountCarry(a_1,\dots,a_{2^{\rho-1}}),\CountSave(a_1,\dots,a_{2^{\rho-1}}),\CountCarry(a_{2^{\rho-1}+1},\dots,a_{2^\rho}),\CountSave(a_{2^{\rho-1}+1},\dots,a_{2^\rho}))$, and
 \item $\Count(a_1,\dots,a_{2^\rho}) = \Add(\CountCarry(a_1,\dots,a_{2^\rho}),\CountSave(a_1,\dots,a_{2^\rho}))$
\end{itemize}
\end{definition}

% AddCarry(as,bs,cs)=zipWith3 xor as bs cs
% AddSave(as,bs,cs)=zipWith3
%
% CountCarry(a_1) = \fff
% CountCarry(as++bs) = AddCarry4(CountCarry(as),CountSave(as),CountCarry(bs),CountSave(bs))
% CountSave(a_1) = a_1
% CountSave(as++bs) = AddSave4(CountCarry(as),CountSave(as),CountCarry(bs),CountSave(bs))
%
% Count(a_1,\dots,a_n) = Add(CountCarry(a_1,\dots,a_n),CountSave(a_1,\dots,a_n))

\begin{lemma}\label{lem:CSAdd4}
Carry-save addition is correct with respect to normal addition
\[
\vlproof{}{}
{
\Add(\AddCarryF(a,b,c,d),\AddSaveF(a,b,c,d))=\Add(\Add(a,b),\Add(c,d))
}
\]
\end{lemma}

\begin{lemma}\label{lem:count-decomp}
\[\vlproof{}{}{\Add(\Count(\vec b_1), \Count(\vec b_2))=\Count(\vec b_1,\vec b_2)}\]
\end{lemma}

All the above definitions and results are only there to prove the following two results (so whatever is not needed can be changed or removed):

\begin{lemma}\label{lem:at-most-full}
$\Count$ is bounded by its number of arguments
\[
\vlproof{}{}{n\geq\Count(a_1,\dots,a_n)}
\quad.\]
\end{lemma}

\begin{proof}
By induction. The base case is:
\[
\vlproof{}{}{1\geq\Count(a)}
\quad.
\]
The inductive case is:
\[
\vlderivation
{
  \vlde{}{}
  {
    n\geq\Count(a_1,\dots,a_n)
  }
  {
    \vlde{}{}
    {
      \Add(n/2,n/2)\geq\Add(\Count(a_1,\dots,a_{n/2}),\Count(a_{n/2+1},\dots,a_n))
    }
    {
      \vlhy
      {
	\vls(n/2\geq\Count(a_1,\dots,a_{n/2}).n/2\geq\Count(a_{n/2+1},\dots,a_n))
      }
    }
  }
}
\quad.
\]
\end{proof}

\begin{lemma}\label{lem:push-comparisons-one-step}
Given two sets of atoms, if the count of the first set is greater than the count of the second set, then either the count of the first half of the first set is greater than the count of the first half of the second set, or the count of the second half of the first set is greater than the count of the second half of the second set.

Given atoms $a_1$, $\dots$, $a_{2^\rho}$ and $b_1$, $\dots$, $b_{2^\rho}$, the following derivation can be constructed:
\[
  \vlder{}{}
  {
    \vls
    [
      \Count(a_1, \dots, a_{2^{\rho-1}})>\Count(b_1, \dots, b_{2^{\rho-1}})
    .
      \Count(a_{2^{\rho-1}+1}, \dots, a_{2^\rho})>\Count(b_{2^{\rho-1}+1}, \dots, b_{2^\rho})
    ]
  }
  {
    \Count(a_1, \dots, a_{2^\rho})>\Count(b_1, \dots, b_{2^\rho})
  }
\quad.\]
\end{lemma}

\begin{proof}
By Lemmas~\ref{lem:basics} and \ref{lem:CSAdd4}:
\[
\vlderivation{
  \vlde{\Phi_3}{}
  {
    \vls
    [
      \Count(a_1, \dots, a_{2^{\rho-1}})>\Count(b_1, \dots, b_{2^{\rho-1}})
    .
      \Count(a_{2^{\rho-1}+1}, \dots, a_{2^\rho})>\Count(b_{2^{\rho-1}+1}, \dots, b_{2^\rho})
    ]
  }
  {
    \vlde{\Phi_2}{}
    {
      \Add(\Count(a_1, \dots, a_{2^{\rho-1}}),\Count(a_{2^{\rho-1}+1}, \dots, a_{2^\rho}))>\Add(\Count(b_1, \dots, b_{2^{\rho-1}}),\Count(b_{2^{\rho-1}+1}, \dots, b_{2^\rho}))
    }
    {
      \vlde{\Phi_1}{}
      {
	\Add(\Count(a_1, \dots, a_{2^{\rho-1}}),\Count(a_{2^{\rho-1}+1}, \dots, a_{2^\rho}))>\Count(b_1, \dots, b_{2^\rho})
      }
      {
	\vlhy
	{
	  \Count(a_1, \dots, a_{2^\rho})>\Count(b_1, \dots, b_{2^\rho})
	}
      }
    }
  }
}
\quad.\]
\begin{itemize}
 \item $\Phi_1$ relies on: $\vlder{}{}{\vls[a_1>b_1.a_2>b_2]}{\Add(a_1,a_2)>\Add(b_1,b_2)}$;
 \item $\Phi_2$ relies on: $\vlder{}{}{\vls(a>b.b=c)}{a>c}\;$ and Lemma~\ref{lem:count-decomp}; and
 \item $\Phi_3$ relies on: $\vlder{}{}{\vls(b>a.b=c)}{c>a}\;$ and Lemma~\ref{lem:count-decomp}.
\end{itemize}
\end{proof}

\begin{definition}
This is the intuitive meaning of the atoms and predicates we will use:
\begin{itemize}
 \item $p_{i,j}$: pigeon $i$ is in hole $j$;
 \item $r_j^m\leftrightarrow\OR{i=0}{m}p_{i,j}$: one of the first $m+1$ pigeons is in hole $j$;
 \item $\Count(r_0^m,\dots,r_{n-1}^m)$: the number of holes inhabited by one of the first $m+1$ pigeons; and
 \item $\PHP{n}$: if there are $n+1$ pigeons and $n$ holes, and every pigeon is in a hole, then there is a hole with two pigeons.
\end{itemize}
\end{definition}


\begin{lemma}\label{lem:push-comparisons}
Select a subset of the holes, by dividing the holes in two $\rho$ times and picking the $t^\text{th}$ subset, so obtained. If adding the $m+2^\text{nd}$ pigeon decreases the number of the selected holes that are inhabited, and at most all of the selected holes are inhabited, then one of the selected holes was inhabited by one of the first $m+1$ pigeons, but not by the first $m+2$.
For $0\leq\rho$, such that $2^\rho<n$ and $0\leq t\leq n$.
\[
  \vlder{}{}
  {
    \OR{j={t2^\rho}}{(t+1)2^\rho-1}\vlsbr(r_j^m.\bar r_j^{m+1})
  }
  {
    \vls(\Count(r_{t2^\rho}^m,\dots,r_{(t+1)2^\rho-1}^m)>\Count(r_{t(2^\rho)}^{m+1},\dots,r_{(t+1)2^\rho-1}^{m+1}).2^\rho>\Count(r_{t2^\rho}^{m+1},\dots,r_{(t+1)2^\rho-1}^{m+1}))
  }
\quad.\]
\end{lemma}

\begin{proof}
By induction and Lemma~\ref{lem:push-comparisons-one-step}.
\end{proof}

\begin{lemma}\label{lem:decrease:evacuation}
If adding the $m+2^\text{nd}$ pigeon decreased the number of inhabited holes,
then there is a hole that was inhabited by one of the first $m+1$ pigeons, but is not inhabited by one of the first $m+2$ pigeons.
\[
  \vlder{}{}
  {
    \OR{j=0}{n-1}\vlsbr(r_j^m.\bar r_j^{m+1})
  }
  {
    \Count(r_0^m,\dots,r_{n-1}^m)>\Count(r_0^{m+1},\dots,r_{n-1}^{m+1})
  }
\quad.\]
\end{lemma}

\begin{proof}
By Lemma~\ref{lem:push-comparisons}.
\end{proof}

\begin{lemma}\label{lem:not-increasing:no-new-hole_subtract}
If adding the $m+2^\text{nd}$ pigeon did not increase the number of inhabited holes,
then either every hole inhabited by one of the $m+2$ first pigeons was also inhabited by one of the $m+1$ first pigeons,
or there is a hole that was inhabited by one of the first $m+1$ pigeons, but is not inhabited by one of the first $m+2$ pigeons.
\[
  \vlder{}{}{\vlsbr[\AND{j=0}{n-1}\vlsbr[r_j^m.\bar r_j^{m+1}].\OR{j=0}{n-1}\vlsbr(r_j^m.\bar r_j^{m+1})]}{\Count(r_0^m,\dots,r_{n-1}^m)\geq\Count(r_0^{m+1},\dots,r_{n-1}^{m+1})}
\quad.\]
\end{lemma}

\begin{proof}
By Lemma~\ref{lem:decrease:evacuation}
\end{proof}

\begin{lemma}\label{lem:no-new-hole:PHP}
If no hole that was not inhabited by the first $m+1$ pigeons is inhabited by the first $m+2$ pigeons, then there is a hole with two pigeons, so the pigeon hole principle holds.
\[
\vlder{}{}{\PHP{n}}{\AND{j=0}{n-1}\vlsbr[r_j^m.\bar r_j^{m+1}]}
\quad.\]
\end{lemma}

\begin{proof}
\[
\vlderivation
{
  \vlde{}{\{\awd\}}
  {
    \PHP{n}
  }
  {
    \vlde{}{\{\swi\}}
    {
      \vls[\OR{j=0}{m}\OR{j=0}{n-1}\vlsbr(p_{i,j}.p_{m+1,j})\;.\;\AND{j=0}{n-1}\bar p_{m+1,j}]
    }
    {
      \vlhy
      {
	\AND{j=0}{n-1}
	\left(
	  \vlder{}{\swi}
	  {
	    \vls[\OR{i=0}{m}\vlsbr(p_{i,j}.p_{m+1,j}).\vlinf{}{}{\bar p_{m+1,j}}{\vls[\bar p_{m+1,j}.\bar p_{m+1,j}]}]
	  }
	  {
	    \vls[\OR{i=0}{m}\vlsbr(p_{i,j}.\vlinf{}{}{\vls[p_{m+1,j}.\bar p_{m+1,j}]}{\ttt}).(\vlinf{}{}{\ttt}{\bar r_j^m}\;.\;\bar p_{m+1,j})]
	  }
	\right)
      }
    }
  }
}
\quad.\]
\end{proof}

\begin{lemma}\label{lem:subtract:contradiction}
For a hole that was inhabited by the first $m+1$ pigeons to not be inhabited by the first $m+2$ pigeons, is a contradiction.
\[
  \vlder{}{}{\fff}{\OR{j=0}{n-1}\vlsbr(r_j^m.\bar r_j^{m+1})}
\quad.\]
\end{lemma}

\begin{proof}
\[
  \OR{j=0}{n-1}\vlsbr(\vlinf{}{}{\fff}{\vls(r_j^m.\bar r_j^m)}\;.\;\vlinf{}{}{\ttt}{\bar p_{m+1,j}})
\quad.\]
\end{proof}

\begin{lemma}\label{lem:not-increasing:PHP}
If adding one pigeon does not increase the number of holes inhabited, then there is a hole with two pigeons, so the pigeon hole principle holds.
\[
\vlder{}{}{\PHP{n}}{\Count(r_0^m,\dots,r_{n-1}^m)\geq\Count(r_0^{m+1},\dots,r_{n-1}^{m+1})}
\quad.\]
\end{lemma}

\begin{proof}
By Lemmas~\ref{lem:no-new-hole:PHP}, \ref{lem:subtract:contradiction} and \ref{lem:not-increasing:no-new-hole_subtract}.
\end{proof}

\begin{lemma}\label{lem:no-pigeon:PHP}
If the first pigeon is not in a hole, then the pigeon hole principle holds.
\[
\vlder{}{}{\PHP{n}}{0\geq\Count(r_0^0,\dots,r_{n-1}^0)}
\quad.\]
\end{lemma}

\begin{lemma}\label{lem:bound:not-increasing_bound}
If at most $m$ holes are inhabeted by the first $m+1$ pigeons, then either adding the $m+1^\text{st}$ pigeon did not increase the number of holes inhabited, or it did, and at most $m-1$ holes are inhabited by the first $m$ pigeons.
\[
\vlder{}{}
{
  \vls
  [
    \Count(r_0^{m-1},\dots,r_{n-1}^{m-1})\geq\Count(r_0^m,\dots,r_{n-1}^m)
  \;.\;
    (m-1)\geq\Count(r_0^{m-1},\dots,r_{n-1}^{m-1})
  ]
}
{
  m\geq\Count(r_0^m,\dots,r_{n-1}^m)
}
\quad.\]
\end{lemma}

\begin{theorem}
Given $n$ pigeons and $n-1$ pigeon holes, if every pigeon is in a hole, then there is a hole with two pigeons.
\[
\vlproof{}{}{\PHP{n}}
\quad.\]
\end{theorem}

\begin{proof}
By Lemmas~\ref{lem:at-most-full}, \ref{lem:no-pigeon:PHP}, \ref{lem:not-increasing:PHP} and \ref{lem:bound:not-increasing_bound}.
\end{proof}


\end{document}

