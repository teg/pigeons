\documentclass[a4paper,10pt,draft]{article}

\usepackage{times}
\usepackage[T1]{fontenc}

\usepackage{a4wide}

\usepackage{ucs}
\usepackage[utf8x]{inputenc}
\usepackage{amsmath}
\usepackage[noxy]{virginialake}

\usepackage[pdftex,pdfborder={0 0 0}]{hyperref}

\usepackage{amsthm}
\theoremstyle{plain}
\newtheorem{theorem}{Theorem}
\newtheorem{lemma}[theorem]{Lemma}
\newtheorem{proposition}[theorem]{Proposition}
\newtheorem{corollary}[theorem]{Corollary}

\theoremstyle{definition}
\newtheorem{definition}[theorem]{Definition}
\newtheorem{remark}[theorem]{Remark}
\newtheorem{notation}[theorem]{Notation}
\newtheorem{para}[theorem]{}

\title{Polynomial Proof of the Pigeon Hole Principle}

\author{Lutz and Tom}

\begin{document}

\maketitle

\newcommand{\fff  }{{\mathsf{f}}}
\newcommand{\ttt  }{{\mathsf{t}}}
\newcommand{\ai   }{{\mathsf{ai}}}
\newcommand{\aw   }{{\mathsf{aw}}}
\newcommand{\ac   }{{\mathsf{ac}}}
\newcommand{\aid  }{{\ai{\downarrow}}}
\newcommand{\awd  }{{\aw{\downarrow}}}
\newcommand{\acd  }{{\ac{\downarrow}}}
\newcommand{\aiu  }{{\ai{\uparrow}}}
\newcommand{\awu  }{{\aw{\uparrow}}}
\newcommand{\acu  }{{\ac{\uparrow}}}
\newcommand{\swi  }{\mathsf{s}}
\newcommand{\med  }{\mathsf{m}}
\newcommand{\asor }{{=_\mathsf{a}{\downarrow}}}
\newcommand{\asand}{{=_\mathsf{a}{\uparrow}}}
\newcommand{\coor }{{=_{\vee\mathsf{c}}}}
\newcommand{\coand}{{=_{\wedge\mathsf{c}}}}
\newcommand{\fffd }{{{=_{\fff}}{\downarrow}}}
\newcommand{\fffu }{{{=_{\fff}}{\uparrow}}}
\newcommand{\tttd }{{{=_{\ttt}}{\downarrow}}}
\newcommand{\tttu }{{{=_{\ttt}}{\uparrow}}}
\newcommand{\tttord }{{{=_{\ttt\vee}}{\downarrow}}}
\newcommand{\fffandd }{{{=_{\fff\wedge}}{\downarrow}}}
\newcommand{\tttoru }{{{=_{\ttt\vee}}{\uparrow}}}
\newcommand{\fffandu }{{{=_{\fff\wedge}}{\uparrow}}}

\newcommand{\AND}[2]{\bigwedge_{#1}^{#2}}
\newcommand{\OR}[2]{\bigvee_{#1}^{#2}}

\newcommand{\Count}[2]{\mathsf{Count}(\overrightarrow{#1}^{#2})}
\newcommand{\PHP}[1]{\mathsf{PHP}_{#1}}

\begin{definition}
This is the intuitive meaning of the atoms and predicates we will use:
\begin{itemize}
 \item $p_{i,j}$: pigeon $i$ is in hole $j$;
 \item $r_j^m\leftrightarrow\OR{i=0}{m}p_{i,j}$: one of the first $m+1$ pigeons is in hole $j$;
 \item $\Count{r}{m}$: the number of holes inhabited by one of the first $m+1$ pigeons; and
 \item $\PHP{n}$: if there are $n+1$ pigeons and $n$ holes, and every pigeon is in a hole, then there is a hole with two pigeons.
\end{itemize}
\end{definition}


\begin{lemma}\label{lem:at-most-full}
At most all the ($n$) holes are inhabited by all the ($n+1$) pigeons.
\[
\vlproof{}{}{n\geq\Count{r}{n+1}}
\]
\end{lemma}

\begin{lemma}\label{lem:no-pigeon:PHP}
If the first pigeon is not in a hole, then the pigeon hole principle holds.
\[
\vlder{}{}{\PHP{n}}{0\geq\Count{r}{0}}
\]
\end{lemma}

\begin{lemma}\label{lem:not-increasing:PHP}
If adding one pigeon does not increase the number of holes inhabited, then there is a hole with two pigeons, so the pigeon hole principle holds.
\[
\vlder{}{}{\PHP{n}}{\Count{r}{m}\geq\Count{r}{m+1}}
\]
\end{lemma}

\begin{proof}
By Lemmas~\ref{lem:not-increasing:no-new-hole_subtract}, \ref{lem:no-new-hole:PHP} and \ref{lem:subtract:contradiction}.
\end{proof}

\begin{lemma}\label{lem:bound:not-increasing_bound}
If at most $m$ holes are inhabeted by the first $m+1$ pigeons, then either adding the $m+1^\text{st}$ pigeon did not increase the number of holes inhabited, or it did, and at most $m-1$ holes are inhabited by the first $m$ pigeons.
\[
\vlder{}{}
{
  \vls
  [
    \Count{r}{m-1}\geq\Count{r}{m}
  \;.\;
    (m-1)\geq\Count{r}{m-1}
  ]
}
{
  m\geq\Count{r}{m}
}
\]
\end{lemma}

\begin{lemma}\label{lem:no-new-hole:PHP}
If no hole that was not inhabited by the first $m+1$ pigeons is inhabited by the first $m+2$ pigeons, then there is a hole with two pigeons, so the pigeon hole principle holds.
\[
\vlder{}{}{\PHP{n}}{\AND{j=0}{n-1}\vlsbr[r_j^m.\bar r_j^{m+1}]}
\]
\end{lemma}

\begin{proof}
\[
\vlderivation
{
  \vlde{}{\{\awd\}}
  {
    \PHP{n}
  }
  {
    \vlde{}{\{\swi\}}
    {
      \vls[\OR{j=0}{m}\OR{j=0}{n-1}\vlsbr(p_{i,j}.p_{m+1,j})\;.\;\AND{j=0}{n-1}\bar p_{m+1,j}]
    }
    {
      \vlhy
      {
	\AND{j=0}{n-1}
	\left(
	  \vlder{}{\swi}
	  {
	    \vls[\OR{i=0}{m}\vlsbr(p_{i,j}.p_{m+1,j}).\vlinf{}{}{\bar p_{m+1,j}}{\vls[\bar p_{m+1,j}.\bar p_{m+1,j}]}]
	  }
	  {
	    \vls[\OR{i=0}{m}\vlsbr(p_{i,j}.\vlinf{}{}{\vls[p_{m+1,j}.\bar p_{m+1,j}]}{\ttt}).(\vlinf{}{}{\ttt}{\bar r_j^m}\;.\;\bar p_{m+1,j})]
	  }
	\right)
      }
    }
  }
}
\]
\end{proof}

\begin{lemma}\label{lem:subtract:contradiction}
For a hole that was inhabited by the first $m+1$ pigeons to not be inhabited by the first $m+2$ pigeons, is a contradiction.
\[
  \vlder{}{}{\fff}{\OR{j=0}{n-1}\vlsbr(r_j^m.\bar r_j^{m+1})}
\]
\end{lemma}

\begin{proof}
\[
  \OR{j=0}{n-1}\vlsbr(\vlinf{}{}{\fff}{\vls(r_j^m.\bar r_j^m)}\;.\;\vlinf{}{}{\ttt}{\bar p_{m+1,j}})
\]
\end{proof}

\begin{lemma}\label{lem:not-increasing:no-new-hole_subtract}
If adding the $m+2^\text{nd}$ pigeon did not increase the number of inhabited holes, then either every hole inhabited by one of the $m+2$ first pigeons was also inhabited by one of the $m+1$ first pigeons, or there is a hole that was inhabited by one of the first $m+1$ pigeons, but is not inhabited by one of the first $m+2$ pigeons.

\[
  \vlder{}{}{\vlsbr[\AND{j=0}{n-1}\vlsbr[r_j^m.\bar r_j^{m+1}].\OR{j=0}{n-1}\vlsbr(r_j^m.\bar r_j^{m+1})]}{\Count{r}{m}\geq\Count{r}{m+1}}
\]
\end{lemma}

\end{document}